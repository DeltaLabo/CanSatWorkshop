\documentclass{article}
\usepackage{amsmath}
\usepackage{amssymb}
\usepackage{siunitx}
\usepackage{float}
\usepackage{tikz}
\def\checkmark{\tikz\fill[scale=0.4](0,.35) -- (.25,0) -- (1,.7) -- (.25,.15) -- cycle;} 
\usepackage{url}
\usepackage[siunitx,american,RPvoltages]{circuitikz}
\ctikzset{capacitors/scale=0.7}
\ctikzset{diodes/scale=0.7}
\usepackage{tabularx}
\newcolumntype{C}{>{\centering\arraybackslash}X}
\renewcommand\tabularxcolumn[1]{m{#1}}% for vertical centering text in X column
\usepackage{tabu}
\usepackage[spanish,es-tabla,activeacute]{babel}
\usepackage{babelbib}
\usepackage{booktabs}
\usepackage{pgfplots}
\usepackage{hyperref}
\hypersetup{colorlinks = true,
            linkcolor = black,
            urlcolor  = blue,
            citecolor = blue,
            anchorcolor = blue}
\usepgfplotslibrary{units, fillbetween} 
\pgfplotsset{compat=1.16}
\usepackage{bm}
\usetikzlibrary{arrows, arrows.meta, shapes, 3d, perspective, positioning}
\renewcommand{\sin}{\sen} %change from sin to sen
\usepackage{bohr}
\setbohr{distribution-method = quantum,insert-missing = true}
\usepackage{elements}
\usepackage{verbatim}
\usetikzlibrary{mindmap,trees,backgrounds}
 
\definecolor{color_mate}{RGB}{255,255,128}
\definecolor{color_plas}{RGB}{255,128,255}
\definecolor{color_text}{RGB}{128,255,255}
\definecolor{color_petr}{RGB}{255,192,192}
\definecolor{color_made}{RGB}{192,255,192}
\definecolor{color_meta}{RGB}{192,192,255}
\usepackage[edges]{forest}
\usepackage{etoolbox}
\usepackage{schemata}
\newcommand\diagram[2]{\schema{\schemabox{#1}}{\schemabox{#2}}}

\title{DeltaLAB CanSat Interface Definition Document}
\author{Kaleb Granados}
\date{February 2025}

\begin{document}

\maketitle
\newpage

\section{General Description}

El Interface Definition Document (IDD) es un documento técnico que proporciona especificaciones detalladas sobre la implementación de una interfase, incluyendo diagramas, esquemas, layouts y pruebas.

El IDD complementa el IRD y el ICD, proporcionando la información técnica necesaria para el diseño, integración y verificación de la interfase. A diferencia del IRD (que define los requisitos) y el ICD (que describe cómo implementarlos), el IDD proporciona los detalles técnicos precisos de la implementación.

\section{Technical Specifications}

\subsection{Mechanical Interface}
\begin{itemize}
    \item Physical dimensions and tolerances
\end{itemize}

\subsection{Electrical Interface}
\begin{itemize}
    \item Voltage, current, and power ratings
\end{itemize}

\subsection{Data Interface}
\begin{itemize}
    \item Communication protocols (I2C, SPI & UART)
\end{itemize}

\subsection{Environmental Constraints}
\begin{itemize}
    \item Operating humidity, temperature and pressure range
\end{itemize}

\section{Verification Procedures of the Interfaces}

\subsection{Mechanical Verification}
\begin{itemize}
    \item Dimensional inspection and tolerances check
\end{itemize}

\subsection{Electrical Verification}
\begin{itemize}
    \item Voltage and current validation under nominal and extreme conditions
\end{itemize}

\subsection{Data Interface Verification}
\begin{itemize}
    \item Latency and error rate measurement
\end{itemize}


\section{Change Management}

\begin{itemize}
    \item Approval process and documentation updates
\end{itemize}


\end{document}
