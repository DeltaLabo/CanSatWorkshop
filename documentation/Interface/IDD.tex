\documentclass{article}
\input{shared/preamble}

\title{DeltaLAB CanSat Interface Definition Document}
\author{Kaleb Granados}
\date{February 2025}

\begin{document}

\maketitle
\newpage

\section{General Description}

El Interface Definition Document (IDD) es un documento técnico que proporciona especificaciones detalladas sobre la implementación de una interfase, incluyendo diagramas, esquemas, layouts y pruebas.

El IDD complementa el IRD y el ICD, proporcionando la información técnica necesaria para el diseño, integración y verificación de la interfase. A diferencia del IRD (que define los requisitos) y el ICD (que describe cómo implementarlos), el IDD proporciona los detalles técnicos precisos de la implementación.

\section{Technical Specifications}

\subsection{Mechanical Interface}
\begin{itemize}
    \item Physical dimensions and tolerances
\end{itemize}

\subsection{Electrical Interface}
\begin{itemize}
    \item Voltage, current, and power ratings
\end{itemize}

\subsection{Data Interface}
\begin{itemize}
    \item Communication protocols (I2C, SPI & UART)
\end{itemize}

\subsection{Environmental Constraints}
\begin{itemize}
    \item Operating humidity, temperature and pressure range
\end{itemize}

\section{Verification Procedures of the Interfaces}

\subsection{Mechanical Verification}
\begin{itemize}
    \item Dimensional inspection and tolerances check
\end{itemize}

\subsection{Electrical Verification}
\begin{itemize}
    \item Voltage and current validation under nominal and extreme conditions
\end{itemize}

\subsection{Data Interface Verification}
\begin{itemize}
    \item Latency and error rate measurement
\end{itemize}


\section{Change Management}

\begin{itemize}
    \item Approval process and documentation updates
\end{itemize}


\end{document}
