\documentclass{article}
\input{shared/preamble}

\title{DeltaLAB CanSat - Project Overview}
\author{XX}
\date{2025}

\begin{document}

\maketitle

\section{Introduction}
Delta Laboratory's CANSAT is an educational tool designed to introduce university students to practical experiences in Systems Engineering. The project consists of an intensive workshop where participants will explore a detailed breakdown of the system into subsystems, alongside a structured methodology based on international standards. The aim is to enable students to deeply understand the technical and managerial implications of such a project, recognizing its scalability to more complex missions such as the development of a CubeSat.

The CANSAT will be capable of measuring atmospheric conditions at its deployment altitude, estimated to be around \textbf{-- meters}. During its descent, at approximately 5 meters above ground, a parachute deployment mechanism will be triggered to reduce landing velocity and minimize damage. The system will be entirely programmed by the students and will incorporate a robust hardware design including microcontrollers, development boards, PCBs, and 3D printing.

Collected data will be transmitted in real time to a local ground station, where both payload measurements and orientation data (attitude) will be displayed and monitored.

The system comprises the following eight subsystems:

\begin{itemize}
    \item \textbf{Structures \& Assembly (S\&A)}
    \item \textbf{Power Distribution System (PDS)}
    \item \textbf{Parachute Deployment Mechanism (PDM)}
    \item \textbf{Energy Storage System (ESS)}
    \item \textbf{On Board Computer and Comms (OBCC)}
    \item \textbf{Attitude Determination System (ADS)}
    \item \textbf{Atmospheric Measurement System (AMS)}
    \item \textbf{Data Processing System (DPS)}
\end{itemize}

All project documentation and development activities are guided by the \textit{NASA Systems Engineering Handbook} and the \textit{INCOSE Systems Engineering Manual}, ensuring a solid foundation for the application of systems engineering principles. This project serves as a tool to engage more people in the space sector through educational yet high-impact initiatives, providing a gateway to further developments in the field.



\end{document}