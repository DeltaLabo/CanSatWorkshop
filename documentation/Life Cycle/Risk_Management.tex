\documentclass{article}
\input{shared/preamble}

\usepackage{geometry}
\usepackage{longtable}
\geometry{a4paper}

\title{DeltaLAB CanSat Risk Management Document}
\author{Kaleb Granados}
\date{February 2025}

\begin{document}

\maketitle
\newpage

\section*{1. Introduction}

\subsection*{Purpose}
This document describes how risks will be managed within the Cansat project to ensure the success of the mission with an appropriate focus on its complexity.

\subsection*{Scope}
Risk management applies to the 8 subsystems of the Cansat (e.g., communication systems, power, temperature control, etc.).

\subsection*{Relation to Other Documents}
This document aligns with the \textit{Systems Engineering Management Plan (SEMP)} and other project planning documents.

\section*{2. Simplified Risk Management Approach}

\subsection*{Risk Management Process}
\begin{itemize}
    \item \textbf{Identification:} Use brainstorming and lessons learned reviews for the 8 subsystems.
    \item \textbf{Analysis:} Qualitative analysis using probability and impact matrices.
    \item \textbf{Mitigation:} Simple strategies such as redundant design for critical subsystems.
\end{itemize}

\subsection*{Required Tools}
\begin{itemize}
    \item Simplified risk matrix.
    \item Risk log in a spreadsheet or lightweight tool like Trello.
\end{itemize}

\section*{3. Roles and Responsibilities}

\subsection*{Subsystem Risk Owners}
Each subsystem owner (e.g., communications, thermal control) will be responsible for identifying and tracking risks related to their area.

\subsection*{Global Risk Manager}
One person will coordinate the risk management process across all subsystems.

\section*{4. Risk Identification}

\subsection*{Risk Sources}
Review of the 8 Cansat subsystems considering:
\begin{itemize}
    \item Hardware failures
    \item Adverse environmental conditions
    \item Communication risks
    \item Power limitations
\end{itemize}

\subsection*{Simplified Methods}
Brainstorming sessions with the engineering team for each subsystem.

\section*{5. Risk Analysis}

\subsection*{Qualitative Analysis}
\begin{itemize}
    \item Probability estimation: low, medium, high
    \item Impact estimation: low, medium, high
\end{itemize}

\subsection*{Quantitative Analysis (if necessary)}
If significant risks are detected, such as signal loss, a more technical approach like signal loss simulations could be used.

\section*{6. Risk Response Planning}

\subsection*{Response Strategies}
\begin{itemize}
    \item \textbf{Avoid:} Improve design to prevent critical component failures.
    \item \textbf{Mitigate:} Implement redundancy in the communication system.
    \item \textbf{Accept:} Minor risks that do not affect mission success.
\end{itemize}

\section*{7. Monitoring and Control}

\subsection*{Periodic Review}
Bi-weekly risk review meetings focused on the most critical subsystems.

\subsection*{Tracking Tools}
Use a spreadsheet or Trello for risk tracking and monitoring.

\section*{8. Tools and Repositories}

\subsection*{Supporting Tools}
Spreadsheets to store identified risks, mitigation actions, and risk owners.

\subsection*{Document Repositories}
A shared document where the status of risks and actions is updated.

\section*{9. Key Metrics and Indicators}

\subsection*{Key Metrics}
\begin{itemize}
    \item Percentage of identified risks mitigated.
    \item Number of critical risks resolved before launch.
\end{itemize}

\subsection*{Lessons Learned Documentation}
Ensure lessons learned are stored in an accessible database for future Cansat teams.

\section*{10. Appendices}

\subsection*{Initial Risk Matrix}
\begin{longtable}{|l|l|l|l|}
    \hline
    \textbf{Risk} & \textbf{Probability} & \textbf{Impact} & \textbf{Response Strategy} \\
    \hline
    \endfirsthead
    \hline
    \endfoot
    Communication failure & High & High & Mitigate with redundancy \\
    Battery failure & Medium & High & Avoid with robust design \\
    Adverse environmental conditions & Low & Medium & Accept, evaluate conditions \\
    Signal loss & Medium & High & Mitigate with pre-launch testing \\
\end{longtable}

\subsection*{Risk Evaluation Form}
\[
\text{Risk:} \quad \text{(Risk description)} \quad \quad \text{Probability:} \quad \text{(Low/Medium/High)}
\]
\[
\text{Impact:} \quad \text{(Low/Medium/High)} \quad \quad \text{Response Strategy:} \quad \text{(Avoid/Transfer/Mitigate/Accept)}
\]

\end{document}